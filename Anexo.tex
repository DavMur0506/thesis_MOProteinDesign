\anexo{}


\chapter{Anexos} \label{chap:anexA}
Los anexos son secciones relativamente independientes que permiten conocer más a fondo aspectos específicos, que por su longitud o naturaleza no conviene incluir dentro del documento principal. Son elementos para dar una información más completa y que es útil para investigaciones futuras. Los anexos constituyen una sección adicional a la organización del trabajo y en ellos debe incluirse material complementario como: estadísticas, gráficas, fotografías, mapas, tablas, programas de cómputo, etcétera. Si no se menciona en el texto principal no deben incluirse. La información de los anexos debe ser completa, de manera que pueda utilizarse de forma independiente.

De acuerdo con las características de la información, el formato es libre. Se recomienda colocarlos en el orden en que están citados en el texto y, de preferencia, usando letras mayúsculas (Anexo A, Anexo B, Anexo C, etc.)

Las imágenes y las tablas que se añadan a los anexos deben de seguir con la numeración del contenido.  
